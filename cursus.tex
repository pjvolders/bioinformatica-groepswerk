% Options for packages loaded elsewhere
\PassOptionsToPackage{unicode}{hyperref}
\PassOptionsToPackage{hyphens}{url}
\PassOptionsToPackage{dvipsnames,svgnames,x11names}{xcolor}
%
\documentclass[
  letterpaper,
  DIV=11,
  numbers=noendperiod]{scrartcl}

\usepackage{amsmath,amssymb}
\usepackage{iftex}
\ifPDFTeX
  \usepackage[T1]{fontenc}
  \usepackage[utf8]{inputenc}
  \usepackage{textcomp} % provide euro and other symbols
\else % if luatex or xetex
  \usepackage{unicode-math}
  \defaultfontfeatures{Scale=MatchLowercase}
  \defaultfontfeatures[\rmfamily]{Ligatures=TeX,Scale=1}
\fi
\usepackage{lmodern}
\ifPDFTeX\else  
    % xetex/luatex font selection
\fi
% Use upquote if available, for straight quotes in verbatim environments
\IfFileExists{upquote.sty}{\usepackage{upquote}}{}
\IfFileExists{microtype.sty}{% use microtype if available
  \usepackage[]{microtype}
  \UseMicrotypeSet[protrusion]{basicmath} % disable protrusion for tt fonts
}{}
\makeatletter
\@ifundefined{KOMAClassName}{% if non-KOMA class
  \IfFileExists{parskip.sty}{%
    \usepackage{parskip}
  }{% else
    \setlength{\parindent}{0pt}
    \setlength{\parskip}{6pt plus 2pt minus 1pt}}
}{% if KOMA class
  \KOMAoptions{parskip=half}}
\makeatother
\usepackage{xcolor}
\setlength{\emergencystretch}{3em} % prevent overfull lines
\setcounter{secnumdepth}{-\maxdimen} % remove section numbering
% Make \paragraph and \subparagraph free-standing
\makeatletter
\ifx\paragraph\undefined\else
  \let\oldparagraph\paragraph
  \renewcommand{\paragraph}{
    \@ifstar
      \xxxParagraphStar
      \xxxParagraphNoStar
  }
  \newcommand{\xxxParagraphStar}[1]{\oldparagraph*{#1}\mbox{}}
  \newcommand{\xxxParagraphNoStar}[1]{\oldparagraph{#1}\mbox{}}
\fi
\ifx\subparagraph\undefined\else
  \let\oldsubparagraph\subparagraph
  \renewcommand{\subparagraph}{
    \@ifstar
      \xxxSubParagraphStar
      \xxxSubParagraphNoStar
  }
  \newcommand{\xxxSubParagraphStar}[1]{\oldsubparagraph*{#1}\mbox{}}
  \newcommand{\xxxSubParagraphNoStar}[1]{\oldsubparagraph{#1}\mbox{}}
\fi
\makeatother

\usepackage{color}
\usepackage{fancyvrb}
\newcommand{\VerbBar}{|}
\newcommand{\VERB}{\Verb[commandchars=\\\{\}]}
\DefineVerbatimEnvironment{Highlighting}{Verbatim}{commandchars=\\\{\}}
% Add ',fontsize=\small' for more characters per line
\usepackage{framed}
\definecolor{shadecolor}{RGB}{241,243,245}
\newenvironment{Shaded}{\begin{snugshade}}{\end{snugshade}}
\newcommand{\AlertTok}[1]{\textcolor[rgb]{0.68,0.00,0.00}{#1}}
\newcommand{\AnnotationTok}[1]{\textcolor[rgb]{0.37,0.37,0.37}{#1}}
\newcommand{\AttributeTok}[1]{\textcolor[rgb]{0.40,0.45,0.13}{#1}}
\newcommand{\BaseNTok}[1]{\textcolor[rgb]{0.68,0.00,0.00}{#1}}
\newcommand{\BuiltInTok}[1]{\textcolor[rgb]{0.00,0.23,0.31}{#1}}
\newcommand{\CharTok}[1]{\textcolor[rgb]{0.13,0.47,0.30}{#1}}
\newcommand{\CommentTok}[1]{\textcolor[rgb]{0.37,0.37,0.37}{#1}}
\newcommand{\CommentVarTok}[1]{\textcolor[rgb]{0.37,0.37,0.37}{\textit{#1}}}
\newcommand{\ConstantTok}[1]{\textcolor[rgb]{0.56,0.35,0.01}{#1}}
\newcommand{\ControlFlowTok}[1]{\textcolor[rgb]{0.00,0.23,0.31}{\textbf{#1}}}
\newcommand{\DataTypeTok}[1]{\textcolor[rgb]{0.68,0.00,0.00}{#1}}
\newcommand{\DecValTok}[1]{\textcolor[rgb]{0.68,0.00,0.00}{#1}}
\newcommand{\DocumentationTok}[1]{\textcolor[rgb]{0.37,0.37,0.37}{\textit{#1}}}
\newcommand{\ErrorTok}[1]{\textcolor[rgb]{0.68,0.00,0.00}{#1}}
\newcommand{\ExtensionTok}[1]{\textcolor[rgb]{0.00,0.23,0.31}{#1}}
\newcommand{\FloatTok}[1]{\textcolor[rgb]{0.68,0.00,0.00}{#1}}
\newcommand{\FunctionTok}[1]{\textcolor[rgb]{0.28,0.35,0.67}{#1}}
\newcommand{\ImportTok}[1]{\textcolor[rgb]{0.00,0.46,0.62}{#1}}
\newcommand{\InformationTok}[1]{\textcolor[rgb]{0.37,0.37,0.37}{#1}}
\newcommand{\KeywordTok}[1]{\textcolor[rgb]{0.00,0.23,0.31}{\textbf{#1}}}
\newcommand{\NormalTok}[1]{\textcolor[rgb]{0.00,0.23,0.31}{#1}}
\newcommand{\OperatorTok}[1]{\textcolor[rgb]{0.37,0.37,0.37}{#1}}
\newcommand{\OtherTok}[1]{\textcolor[rgb]{0.00,0.23,0.31}{#1}}
\newcommand{\PreprocessorTok}[1]{\textcolor[rgb]{0.68,0.00,0.00}{#1}}
\newcommand{\RegionMarkerTok}[1]{\textcolor[rgb]{0.00,0.23,0.31}{#1}}
\newcommand{\SpecialCharTok}[1]{\textcolor[rgb]{0.37,0.37,0.37}{#1}}
\newcommand{\SpecialStringTok}[1]{\textcolor[rgb]{0.13,0.47,0.30}{#1}}
\newcommand{\StringTok}[1]{\textcolor[rgb]{0.13,0.47,0.30}{#1}}
\newcommand{\VariableTok}[1]{\textcolor[rgb]{0.07,0.07,0.07}{#1}}
\newcommand{\VerbatimStringTok}[1]{\textcolor[rgb]{0.13,0.47,0.30}{#1}}
\newcommand{\WarningTok}[1]{\textcolor[rgb]{0.37,0.37,0.37}{\textit{#1}}}

\providecommand{\tightlist}{%
  \setlength{\itemsep}{0pt}\setlength{\parskip}{0pt}}\usepackage{longtable,booktabs,array}
\usepackage{calc} % for calculating minipage widths
% Correct order of tables after \paragraph or \subparagraph
\usepackage{etoolbox}
\makeatletter
\patchcmd\longtable{\par}{\if@noskipsec\mbox{}\fi\par}{}{}
\makeatother
% Allow footnotes in longtable head/foot
\IfFileExists{footnotehyper.sty}{\usepackage{footnotehyper}}{\usepackage{footnote}}
\makesavenoteenv{longtable}
\usepackage{graphicx}
\makeatletter
\def\maxwidth{\ifdim\Gin@nat@width>\linewidth\linewidth\else\Gin@nat@width\fi}
\def\maxheight{\ifdim\Gin@nat@height>\textheight\textheight\else\Gin@nat@height\fi}
\makeatother
% Scale images if necessary, so that they will not overflow the page
% margins by default, and it is still possible to overwrite the defaults
% using explicit options in \includegraphics[width, height, ...]{}
\setkeys{Gin}{width=\maxwidth,height=\maxheight,keepaspectratio}
% Set default figure placement to htbp
\makeatletter
\def\fps@figure{htbp}
\makeatother

\KOMAoption{captions}{tableheading}
\makeatletter
\@ifpackageloaded{caption}{}{\usepackage{caption}}
\AtBeginDocument{%
\ifdefined\contentsname
  \renewcommand*\contentsname{Table of contents}
\else
  \newcommand\contentsname{Table of contents}
\fi
\ifdefined\listfigurename
  \renewcommand*\listfigurename{List of Figures}
\else
  \newcommand\listfigurename{List of Figures}
\fi
\ifdefined\listtablename
  \renewcommand*\listtablename{List of Tables}
\else
  \newcommand\listtablename{List of Tables}
\fi
\ifdefined\figurename
  \renewcommand*\figurename{Figure}
\else
  \newcommand\figurename{Figure}
\fi
\ifdefined\tablename
  \renewcommand*\tablename{Table}
\else
  \newcommand\tablename{Table}
\fi
}
\@ifpackageloaded{float}{}{\usepackage{float}}
\floatstyle{ruled}
\@ifundefined{c@chapter}{\newfloat{codelisting}{h}{lop}}{\newfloat{codelisting}{h}{lop}[chapter]}
\floatname{codelisting}{Listing}
\newcommand*\listoflistings{\listof{codelisting}{List of Listings}}
\makeatother
\makeatletter
\makeatother
\makeatletter
\@ifpackageloaded{caption}{}{\usepackage{caption}}
\@ifpackageloaded{subcaption}{}{\usepackage{subcaption}}
\makeatother

\ifLuaTeX
  \usepackage{selnolig}  % disable illegal ligatures
\fi
\usepackage{bookmark}

\IfFileExists{xurl.sty}{\usepackage{xurl}}{} % add URL line breaks if available
\urlstyle{same} % disable monospaced font for URLs
\hypersetup{
  colorlinks=true,
  linkcolor={blue},
  filecolor={Maroon},
  citecolor={Blue},
  urlcolor={Blue},
  pdfcreator={LaTeX via pandoc}}


\author{}
\date{}

\begin{document}


\section{Bio-informatica groepswerk}\label{bio-informatica-groepswerk}

\subsection{Inhoudsopgave}\label{inhoudsopgave}

\begin{enumerate}
\def\labelenumi{\arabic{enumi}.}
\tightlist
\item
  Introductie tot Linux Shell/Bash
\item
  Kwaliteitscontrole met FastQC
\item
  Lees Mapping met BWA
\item
  Variant Calling met BCFtools
\item
  Variant Filtering
\item
  Variant Annotatie met SnpEff
\item
  Simuleren van FASTQ-bestanden met NEAT
\end{enumerate}

\subsection{1. Introductie tot Linux
Shell/Bash}\label{introductie-tot-linux-shellbash}

\subsubsection{Wat is een Shell?}\label{wat-is-een-shell}

Een shell is een programma dat een interface biedt voor gebruikers om
met het besturingssysteem te communiceren. De meest voorkomende shell in
Linux-systemen heet Bash (Bourne Again SHell). Wanneer je de
opdrachtregel gebruikt, typ je opdrachten in de shell.

\subsubsection{Basisbegrippen}\label{basisbegrippen}

\begin{enumerate}
\def\labelenumi{\arabic{enumi}.}
\item
  \textbf{Opdrachtprompt}: Hier typ je je opdrachten. Het eindigt
  meestal met een \texttt{\$}-teken.
\item
  \textbf{Opdrachten}: Dit zijn instructies die je aan de computer
  geeft.
\item
  \textbf{Argumenten}: Aanvullende informatie die je aan een opdracht
  geeft.
\item
  \textbf{Opties}: Wijzigen het gedrag van opdrachten, meestal beginnend
  met een streepje (\texttt{-}).
\end{enumerate}

\subsubsection{Basisopdrachten}\label{basisopdrachten}

\paragraph{\texorpdfstring{1. \texttt{pwd} (Print Working
Directory)}{1. pwd (Print Working Directory)}}\label{pwd-print-working-directory}

Toont je huidige locatie in het bestandssysteem.

\begin{Shaded}
\begin{Highlighting}[]
\ExtensionTok{$}\NormalTok{ pwd}
\ExtensionTok{/home/gebruikersnaam}
\end{Highlighting}
\end{Shaded}

\paragraph{\texorpdfstring{2. \texttt{ls}
(List)}{2. ls (List)}}\label{ls-list}

Geeft een lijst van bestanden en mappen in de huidige directory.

\begin{Shaded}
\begin{Highlighting}[]
\ExtensionTok{$}\NormalTok{ ls}
\ExtensionTok{Documenten}\NormalTok{  Downloads  Afbeeldingen  Muziek}
\end{Highlighting}
\end{Shaded}

Opties: - \texttt{ls\ -l}: Lang formaat, toont meer details -
\texttt{ls\ -a}: Toont verborgen bestanden (die beginnen met een punt)

\paragraph{\texorpdfstring{3. \texttt{cd} (Change
Directory)}{3. cd (Change Directory)}}\label{cd-change-directory}

Verplaatst je naar een andere directory.

\begin{Shaded}
\begin{Highlighting}[]
\ExtensionTok{$}\NormalTok{ cd Documenten}
\end{Highlighting}
\end{Shaded}

Speciale directories: - \texttt{.} : Huidige directory - \texttt{..}:
Bovenliggende directory - \texttt{\textasciitilde{}} : Thuisdirectory

\paragraph{\texorpdfstring{4. \texttt{mkdir} (Make
Directory)}{4. mkdir (Make Directory)}}\label{mkdir-make-directory}

Maakt een nieuwe directory aan.

\begin{Shaded}
\begin{Highlighting}[]
\ExtensionTok{$}\NormalTok{ mkdir NieuweMap}
\end{Highlighting}
\end{Shaded}

\paragraph{\texorpdfstring{5. \texttt{cp}
(Copy)}{5. cp (Copy)}}\label{cp-copy}

Kopieert bestanden of directories.

\begin{Shaded}
\begin{Highlighting}[]
\ExtensionTok{$}\NormalTok{ cp bestand.txt Documenten/}
\end{Highlighting}
\end{Shaded}

Om een directory en zijn inhoud te kopiëren, gebruik de \texttt{-r}
(recursief) optie:

\begin{Shaded}
\begin{Highlighting}[]
\ExtensionTok{$}\NormalTok{ cp }\AttributeTok{{-}r}\NormalTok{ MapA MapB}
\end{Highlighting}
\end{Shaded}

\paragraph{\texorpdfstring{6. \texttt{mv}
(Move)}{6. mv (Move)}}\label{mv-move}

Verplaatst of hernoemt bestanden en directories.

\begin{Shaded}
\begin{Highlighting}[]
\ExtensionTok{$}\NormalTok{ mv bestand.txt Documenten/}
\ExtensionTok{$}\NormalTok{ mv oudenaam.txt nieuwenaam.txt}
\end{Highlighting}
\end{Shaded}

\paragraph{\texorpdfstring{7. \texttt{rm}
(Remove)}{7. rm (Remove)}}\label{rm-remove}

Verwijdert bestanden of directories. Wees voorzichtig met deze opdracht!

\begin{Shaded}
\begin{Highlighting}[]
\ExtensionTok{$}\NormalTok{ rm bestand.txt}
\end{Highlighting}
\end{Shaded}

Om een directory en zijn inhoud te verwijderen, gebruik de \texttt{-r}
optie:

\begin{Shaded}
\begin{Highlighting}[]
\ExtensionTok{$}\NormalTok{ rm }\AttributeTok{{-}r}\NormalTok{ MapNaam}
\end{Highlighting}
\end{Shaded}

\paragraph{\texorpdfstring{8. \texttt{cat}
(Concatenate)}{8. cat (Concatenate)}}\label{cat-concatenate}

Toont de inhoud van een bestand.

\begin{Shaded}
\begin{Highlighting}[]
\ExtensionTok{$}\NormalTok{ cat bestand.txt}
\end{Highlighting}
\end{Shaded}

\paragraph{\texorpdfstring{9. \texttt{echo}}{9. echo}}\label{echo}

Print tekst naar het scherm.

\begin{Shaded}
\begin{Highlighting}[]
\ExtensionTok{$}\NormalTok{ echo }\StringTok{"Hallo, Wereld!"}
\ExtensionTok{Hallo,}\NormalTok{ Wereld!}
\end{Highlighting}
\end{Shaded}

\subsubsection{Opdrachtstructuur}\label{opdrachtstructuur}

De meeste opdrachten volgen deze structuur:

\begin{verbatim}
opdracht [opties] [argumenten]
\end{verbatim}

Bijvoorbeeld:

\begin{Shaded}
\begin{Highlighting}[]
\ExtensionTok{$}\NormalTok{ ls }\AttributeTok{{-}l}\NormalTok{ Documenten}
\end{Highlighting}
\end{Shaded}

Hier is \texttt{ls} de opdracht, \texttt{-l} een optie, en
\texttt{Documenten} een argument.

\subsubsection{Tips}\label{tips}

\begin{enumerate}
\def\labelenumi{\arabic{enumi}.}
\tightlist
\item
  Gebruik de pijltjestoetsen omhoog en omlaag om door je
  opdrachtgeschiedenis te navigeren.
\item
  Gebruik Tab voor automatische aanvulling van bestands- en mapnamen.
\item
  Gebruik \texttt{man} gevolgd door een opdrachtnaam om de handleiding
  te zien (bijv. \texttt{man\ ls}).
\end{enumerate}

\subsubsection{Oefenopdrachten}\label{oefenopdrachten}

\begin{enumerate}
\def\labelenumi{\arabic{enumi}.}
\tightlist
\item
  Maak een directory genaamd ``BioinformaticaCursus'' in je
  thuisdirectory.
\item
  Maak binnen ``BioinformaticaCursus'' drie subdirectories: ``Data'',
  ``Scripts'' en ``Resultaten''.
\item
  Maak een leeg bestand genaamd ``notities.txt'' in de
  ``BioinformaticaCursus'' directory.
\item
  Toon de inhoud van ``BioinformaticaCursus'' in lang formaat.
\item
  Verplaats ``notities.txt'' naar de ``Resultaten'' directory.
\item
  Kopieer ``notities.txt'' van ``Resultaten'' naar ``Data''.
\item
  Verwijder het ``notities.txt'' bestand uit de ``Data'' directory.
\end{enumerate}

\subsection{2. Kwaliteitscontrole met
FastQC}\label{kwaliteitscontrole-met-fastqc}

\subsubsection{Doel}\label{doel}

Het hoofddoel van FastQC is om een snelle kwaliteitscontrole uit te
voeren op ruwe sequentiedata afkomstig van high-throughput sequencing
pijplijnen. Het helpt bij het identificeren van problemen die kunnen
voortkomen uit de sequencer zelf of de bibliotheekvoorbereiding.

\subsubsection{Opdrachten}\label{opdrachten}

\begin{Shaded}
\begin{Highlighting}[]
\CommentTok{\# Voer FastQC uit op beide read{-}bestanden}
\ExtensionTok{fastqc}\NormalTok{ monster1\_R1.fastq.gz monster1\_R2.fastq.gz}
\end{Highlighting}
\end{Shaded}

\subsubsection{Het interpreteren van het FastQC
HTML-rapport}\label{het-interpreteren-van-het-fastqc-html-rapport}

\begin{enumerate}
\def\labelenumi{\arabic{enumi}.}
\item
  \textbf{Basisstatistieken}: Geeft een overzicht van het bestand,
  waaronder totaal aantal sequenties, sequentielengte en GC-gehalte.
\item
  \textbf{Per base sequentiekwaliteit}: Toont de kwaliteitsscores voor
  elke positie in de read.

  \begin{itemize}
  \tightlist
  \item
    Groen gebied: Goede kwaliteit
  \item
    Oranje gebied: Redelijke kwaliteit
  \item
    Rood gebied: Slechte kwaliteit
  \end{itemize}
\item
  \textbf{Per tegelsequentiekwaliteit}: Relevant voor
  Illumina-sequencers, toont kwaliteit per tegel.
\item
  \textbf{Per sequentiekwaliteitsscores}: Geeft de verdeling van
  kwaliteitsscores over alle sequenties.
\item
  \textbf{Per base sequentie-inhoud}: Toont de verhoudingen van basen op
  elke positie.
\item
  \textbf{Per sequentie GC-inhoud}: Vergelijkt de waargenomen
  GC-inhoudverdeling met een theoretische normale verdeling.
\item
  \textbf{Per base N-inhoud}: Toont het percentage van basen op elke
  positie die niet konden worden bepaald (N).
\item
  \textbf{Sequentielengteverspreiding}: Voor de meeste platformen zou
  dit een scherpe piek moeten zijn.
\item
  \textbf{Sequentieduplicatieniveaus}: Hoge duplicatieniveaus kunnen
  duiden op PCR-bias.
\item
  \textbf{Overgerepresenteerde sequenties}: Lijst van sequenties die
  vaker voorkomen dan verwacht.
\item
  \textbf{Adapter-inhoud}: Toont de aanwezigheid van vaak gebruikte
  adapters in je bibliotheek.
\end{enumerate}

\subsubsection{Opdrachtvragen:}\label{opdrachtvragen}

\begin{enumerate}
\def\labelenumi{\arabic{enumi}.}
\tightlist
\item
  Wat is de gemiddelde kwaliteitsscore over alle basen?
\item
  Zijn er overgerepresenteerde sequenties? Zo ja, wat zouden deze kunnen
  zijn?
\item
  Hoe verandert de kwaliteitsscore over de lengte van de reads?
\end{enumerate}

\subsection{3. Read Mapping met BWA}\label{read-mapping-met-bwa}

BWA (Burrows-Wheeler Aligner) wordt gebruikt om de reads uit te lijnen
tegen een referentiegenoom.

\begin{Shaded}
\begin{Highlighting}[]
\CommentTok{\# Indexeer het referentiegenoom}
\ExtensionTok{bwa}\NormalTok{ index referentiegenoom.fasta}

\CommentTok{\# Lijn reads uit tegen het referentiegenoom}
\ExtensionTok{bwa}\NormalTok{ mem referentiegenoom.fasta monster1\_R1.fastq.gz monster1\_R2.fastq.gz }\OperatorTok{\textgreater{}}\NormalTok{ monster1.sam}
\end{Highlighting}
\end{Shaded}

\subsection{4. Variant Calling met
BCFtools}\label{variant-calling-met-bcftools}

BCFtools wordt gebruikt om varianten te identificeren uit de uitgelijnde
reads.

\begin{Shaded}
\begin{Highlighting}[]
\CommentTok{\# Roep varianten aan}
\ExtensionTok{bcftools}\NormalTok{ mpileup }\AttributeTok{{-}f}\NormalTok{ referentiegenoom.fasta monster1.sorted.bam }\KeywordTok{|} \ExtensionTok{bcftools}\NormalTok{ call }\AttributeTok{{-}mv} \AttributeTok{{-}Ob} \AttributeTok{{-}o}\NormalTok{ monster1.raw.bcf}
\end{Highlighting}
\end{Shaded}

\subsection{5. Variant Filtering}\label{variant-filtering}

VCFtools kan worden gebruikt om de varianten te filteren op basis van
verschillende criteria.

\begin{Shaded}
\begin{Highlighting}[]
\CommentTok{\# Zet BCF om naar VCF}
\ExtensionTok{bcftools}\NormalTok{ view monster1.raw.bcf }\OperatorTok{\textgreater{}}\NormalTok{ monster1.raw.vcf}

\CommentTok{\# Filter varianten}
\ExtensionTok{vcftools} \AttributeTok{{-}{-}vcf}\NormalTok{ monster1.raw.vcf }\DataTypeTok{\textbackslash{}}
         \AttributeTok{{-}{-}minQ}\NormalTok{ 30 }\DataTypeTok{\textbackslash{}}
         \AttributeTok{{-}{-}min{-}meanDP}\NormalTok{ 10 }\DataTypeTok{\textbackslash{}}
         \AttributeTok{{-}{-}max{-}missing}\NormalTok{ 0.8 }\DataTypeTok{\textbackslash{}}
         \AttributeTok{{-}{-}recode} \AttributeTok{{-}{-}recode{-}INFO{-}all} \DataTypeTok{\textbackslash{}}
         \AttributeTok{{-}{-}out}\NormalTok{ monster1.gefilterd}
\end{Highlighting}
\end{Shaded}

\subsection{6. Variant Annotatie met
SnpEff}\label{variant-annotatie-met-snpeff}

\subsubsection{Stap 1: Installeer
SnpEff}\label{stap-1-installeer-snpeff}

\begin{Shaded}
\begin{Highlighting}[]
\FunctionTok{wget}\NormalTok{ https://snpeff.blob.core.windows.net/versions/snpEff\_latest\_core.zip}
\FunctionTok{unzip}\NormalTok{ snpEff\_latest\_core.zip}
\end{Highlighting}
\end{Shaded}

\subsubsection{Stap 2: Download de Human Genome
Database}\label{stap-2-download-de-human-genome-database}

\begin{Shaded}
\begin{Highlighting}[]
\ExtensionTok{java} \AttributeTok{{-}jar}\NormalTok{ snpEff.jar download }\AttributeTok{{-}v}\NormalTok{ hg38}
\end{Highlighting}
\end{Shaded}

\subsubsection{Stap 3: Voer SnpEff uit}\label{stap-3-voer-snpeff-uit}

\begin{Shaded}
\begin{Highlighting}[]
\ExtensionTok{java} \AttributeTok{{-}Xmx4g} \AttributeTok{{-}jar}\NormalTok{ snpEff.jar hg38 input\_varianten.vcf }\OperatorTok{\textgreater{}}\NormalTok{ geannoteerde\_varianten.vcf}
\end{Highlighting}
\end{Shaded}

\subsubsection{Stap 4: Interpreteer de
Resultaten}\label{stap-4-interpreteer-de-resultaten}

Kijk naar het \texttt{ANN} veld in de output VCF voor gedetailleerde
annotaties.

\subsubsection{Opdrachtvragen}\label{opdrachtvragen-1}

\begin{enumerate}
\def\labelenumi{\arabic{enumi}.}
\tightlist
\item
  Hoeveel varianten met HOGE impact heb je gevonden? Wat voor soort
  varianten zijn dit?
\item
  Vind een missense variant. Wat is de aminozuurverandering? In welk gen
  kwam het voor?
\item
  Zijn er varianten in bekende ziekte-geassocieerde genen?
\item
  Wat is het meest voorkomende type variant in je dataset?
\item
  Kun je varianten vinden die de eiwitfunctie kunnen beïnvloeden? Leg
  uit waarom je denkt dat ze impactvol kunnen zijn.
\end{enumerate}

\subsection{7. Simuleren van FASTQ-bestanden met
NEAT}\label{simuleren-van-fastq-bestanden-met-neat}

\subsubsection{Stap 1: Incorporeer Varianten in
Referentiegenoom}\label{stap-1-incorporeer-varianten-in-referentiegenoom}

\begin{Shaded}
\begin{Highlighting}[]
\CommentTok{\# Installeer NEAT (indien nog niet geïnstalleerd)}
\FunctionTok{git}\NormalTok{ clone https://github.com/ncbi/NEAT.git}
\BuiltInTok{cd}\NormalTok{ NEAT}
\FunctionTok{make}

\CommentTok{\# Gebruik NEAT om varianten te incorporeren}
\ExtensionTok{./neat{-}genreads} \DataTypeTok{\textbackslash{}}
  \AttributeTok{{-}rc}\NormalTok{ referentiegenoom.fa }\DataTypeTok{\textbackslash{}}
  \AttributeTok{{-}dv}\NormalTok{ varianten.vcf }\DataTypeTok{\textbackslash{}}
  \AttributeTok{{-}o}\NormalTok{ gemuteerd\_genoom}
\end{Highlighting}
\end{Shaded}

\subsubsection{Stap 2: Genereer
FASTQ-bestanden}\label{stap-2-genereer-fastq-bestanden}

\begin{Shaded}
\begin{Highlighting}[]
\CommentTok{\# Gebruik NEAT om FASTQ{-}bestanden te genereren}
\ExtensionTok{neat{-}genreads} \DataTypeTok{\textbackslash{}}
  \AttributeTok{{-}rf}\NormalTok{ gemuteerd\_genoom.fa }\DataTypeTok{\textbackslash{}}
  \AttributeTok{{-}R}\NormalTok{ 150 }\DataTypeTok{\textbackslash{}}
  \AttributeTok{{-}c}\NormalTok{ 30 }\DataTypeTok{\textbackslash{}}
  \AttributeTok{{-}o}\NormalTok{ gesimuleerde\_reads}
\end{Highlighting}
\end{Shaded}

Dit zal paired-end reads genereren met een lengte van 150 basen en 30x
dekking.

\subsubsection{Python Script voor Meerdere
Monsters}\label{python-script-voor-meerdere-monsters}

Hier is een Python-script om FASTQ-bestanden voor meerdere monsters te
genereren, elk met hun eigen set varianten:

\begin{Shaded}
\begin{Highlighting}[]
\ImportTok{import}\NormalTok{ os}
\ImportTok{import}\NormalTok{ subprocess}
\ImportTok{import}\NormalTok{ argparse}

\KeywordTok{def}\NormalTok{ run\_neat(reference, vcf, output\_prefix, read\_length, coverage):}
\NormalTok{    command }\OperatorTok{=}\NormalTok{ [}
        \StringTok{"neat{-}genreads"}\NormalTok{,}
        \StringTok{"{-}rc"}\NormalTok{, reference,}
        \StringTok{"{-}dv"}\NormalTok{, vcf,}
        \StringTok{"{-}o"}\NormalTok{, output\_prefix,}
        \StringTok{"{-}rl"}\NormalTok{, }\BuiltInTok{str}\NormalTok{(read\_length),}
        \StringTok{"{-}c"}\NormalTok{, }\BuiltInTok{str}\NormalTok{(coverage),}
        \StringTok{"{-}pe"}\NormalTok{, }\StringTok{"150"}\NormalTok{, }\StringTok{"50"}\NormalTok{,  }\CommentTok{\# Paired{-}end reads, 150bp lengte, 50bp stddev}
        \StringTok{"{-}bq"}\NormalTok{, }\StringTok{"30"}  \CommentTok{\# Basiskwaliteitsscore}
\NormalTok{    ]}
\NormalTok{    subprocess.run(command, check}\OperatorTok{=}\VariableTok{True}\NormalTok{)}

\KeywordTok{def}\NormalTok{ main():}
\NormalTok{    parser }\OperatorTok{=}\NormalTok{ argparse.ArgumentParser(description}\OperatorTok{=}\StringTok{"Genereer FASTQ{-}bestanden voor meerdere monsters met NEAT."}\NormalTok{)}
\NormalTok{    parser.add\_argument(}\StringTok{"reference"}\NormalTok{, }\BuiltInTok{help}\OperatorTok{=}\StringTok{"Pad naar het referentiegenoombestand"}\NormalTok{)}
\NormalTok{    parser.add\_argument(}\StringTok{"vcf\_dir"}\NormalTok{, }\BuiltInTok{help}\OperatorTok{=}\StringTok{"Directory met VCF{-}bestanden"}\NormalTok{)}
\NormalTok{    parser.add\_argument(}\StringTok{"output\_dir"}\NormalTok{, }\BuiltInTok{help}\OperatorTok{=}\StringTok{"Directory voor output FASTQ{-}bestanden"}\NormalTok{)}
\NormalTok{    parser.add\_argument(}\StringTok{"{-}{-}read\_length"}\NormalTok{, }\BuiltInTok{type}\OperatorTok{=}\BuiltInTok{int}\NormalTok{, default}\OperatorTok{=}\DecValTok{150}\NormalTok{, }\BuiltInTok{help}\OperatorTok{=}\StringTok{"Leeslengte"}\NormalTok{)}
\NormalTok{    parser.add\_argument(}\StringTok{"{-}{-}coverage"}\NormalTok{, }\BuiltInTok{type}\OperatorTok{=}\BuiltInTok{int}\NormalTok{, default}\OperatorTok{=}\DecValTok{30}\NormalTok{, }\BuiltInTok{help}\OperatorTok{=}\StringTok{"Dekkingsdiepte"}\NormalTok{)}

\NormalTok{    args }\OperatorTok{=}\NormalTok{ parser.parse\_args()}

\NormalTok{    os.makedirs(args.output\_dir, exist\_ok}\OperatorTok{=}\VariableTok{True}\NormalTok{)}

    \ControlFlowTok{for}\NormalTok{ vcf\_file }\KeywordTok{in}\NormalTok{ os.listdir(args.vcf\_dir):}
        \ControlFlowTok{if}\NormalTok{ vcf\_file.endswith(}\StringTok{".vcf"}\NormalTok{):}
\NormalTok{            sample\_name }\OperatorTok{=}\NormalTok{ os.path.splitext(vcf\_file)[}\DecValTok{0}\NormalTok{]}
\NormalTok{            vcf\_path }\OperatorTok{=}\NormalTok{ os.path.join(args.vcf\_dir, vcf\_file)}
\NormalTok{            output\_prefix }\OperatorTok{=}\NormalTok{ os.path.join(args.output\_dir, sample\_name)}

            \BuiltInTok{print}\NormalTok{(}\SpecialStringTok{f"FASTQ genereren voor monster: }\SpecialCharTok{\{}\NormalTok{sample\_name}\SpecialCharTok{\}}\SpecialStringTok{"}\NormalTok{)}
\NormalTok{            run\_neat(args.reference, vcf\_path, output\_prefix, args.read\_length, args.coverage)}

    \BuiltInTok{print}\NormalTok{(}\StringTok{"Alle monsters zijn succesvol verwerkt!"}\NormalTok{)}

\ControlFlowTok{if} \VariableTok{\_\_name\_\_} \OperatorTok{==} \StringTok{"\_\_main\_\_"}\NormalTok{:}
\NormalTok{    main()}
\end{Highlighting}
\end{Shaded}

Gebruik dit script als volgt:

\begin{Shaded}
\begin{Highlighting}[]
\ExtensionTok{python}\NormalTok{ genereer\_multi\_monster\_fastq.py /pad/naar/referentie.fa /pad/naar/vcf\_directory /pad/naar/output\_directory}
\end{Highlighting}
\end{Shaded}

\subsection{Conclusie}\label{conclusie}

Deze cursus biedt een uitgebreide introductie tot bioinformatica, van
basiscommando's in Linux tot geavanceerde onderwerpen zoals
variant-annotatie en het simuleren van sequentiedata. Door deze stappen
te volgen en de opdrachten uit te voeren, zul je een solide basis leggen
voor verder werk in bioinformatica.




\end{document}
